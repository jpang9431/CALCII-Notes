\documentclass[17pt]{extarticle}

% Language setting
% Replace `english' with e.g. `spanish' to change the document language
\usepackage[english]{babel}

% Set page size and margins
% Replace `letterpaper' with `a4paper' for UK/EU standard size
\usepackage[letterpaper,top=2cm,bottom=2cm,left=3cm,right=3cm,marginparwidth=1.75cm]{geometry}

% Useful packages
\usepackage{amsmath}
\usepackage{graphicx}
\usepackage{float}
\usepackage[colorlinks=true, allcolors=blue]{hyperref}
\usepackage{tabularray}
\usepackage{xcolor}
\usepackage{colortbl}
\usepackage{fontspec}



\definecolor{githubbackground}{HTML}{212830}
\definecolor{githublightgray}{HTML}{313844}


\defaultfontfeatures{Ligatures=TeX,Color=white}
\setmainfont{Linux Libertine O}
\everymath{\color{white}}
\pagecolor{githubbackground}

\title{Calc II Notes}
\author{SuperExago and Tepoys}

\begin{document}

\maketitle

\begin{abstract}
This will primarily consist of formulas necessary for calculations
\end{abstract}
\clearpage
\section{Derivatives}
Instantaneous rate of change at a point
\subsection{Limit Definition of Derivatives}
$f\prime(x)=\lim_{h \to 0}{\frac{f(x+h)-f(x)}{h}}$\\
$f\prime(a)=\lim_{x \to a}{\frac{f(x)-f(a)}{x-a}}$

\arrayrulecolor{lightgray}
\subsection{Basic Derivatives}
\renewcommand{\arraystretch}{2}
\begin{table}[H]
\rowcolors{2}{githublightgray}{githubbackground}
\begin{tabular}{|l|l|l|}
\hline
\rowcolor{githublightgray}
Name & Derivative & Result\\

\hline
Constant&$\frac{d}{dx}[C]$&0\\
\hline
Power&$\frac{d}{dx}[u^n]$&$n*u^{n-1}*u\prime$\\
\hline
Exponential&$\frac{d}{dx}[a^u]$&$a^u*u\prime *ln(a)$\\
\hline
Constant Multiply&$\frac{d}{dx}[C*f(x)]$&$C*\frac{d}{dx}[f(x)]$\\
\hline
Multiply Functions&$\frac{d}{dx}[f(x)g(x)]$&$f\prime(x)g(x)+g\prime(x)f(x)$\\
\hline
Divide Functions&$\frac{d}{dx}[\frac{f(x)}{g(x)}]$&$\frac{f\prime(x)g(x)-g\prime(x)f(x)}{g(x)^2}$\\
\hline
Chain Rule&$\frac{d}{dx}[f(g(u))]$&$f\prime(g(u))*g\prime(u)*u\prime$\\
\hline
Logarithm&$\frac{d}{dx}[log_au]$&$\frac{u\prime}{uln(a)}$\\
\hline
\end{tabular}
\end{table}

\arrayrulecolor{lightgray}
\subsection{Trigonometric Derivatives}
\renewcommand{\arraystretch}{2}
\arrayrulecolor{lightgray}
\begin{table}[H]
\rowcolors{2}{githublightgray}{githubbackground}
\begin{tabular}{|l|l|l|}
\hline
\rowcolor{githublightgray}
Name & Derivative & Result\\
\hline
Sine&$\frac{d}{dx}[sin(u)]$&$cos(u)u\prime$\\
\hline
Cosine&$\frac{d}{dx}[cos(u)]$&$-sin(u)u\prime$\\
\hline
Tangent&$\frac{d}{dx}[tan(u)]$&$sec^2(u)u\prime$\\
\hline
Cotangent&$\frac{d}{dx}[cot(u)]$&$-csc^2(u)u\prime$\\
\hline
Secant&$\frac{d}{dx}[sec(u)]$&$sec(u)tan(u)u\prime$\\
\hline
Cosecant&$\frac{d}{dx}[csc(u)]$&$-csc(u)cot(u)u\prime$\\
\hline
\end{tabular}
\end{table}

\subsection{Inverse Trigonometric Derivatives}
For any ``co-'' version of the trig function multiply the \\result by -1
\renewcommand{\arraystretch}{2}
\arrayrulecolor{lightgray}
\begin{table}[H]
\rowcolors{2}{githublightgray}{githubbackground}
\begin{tabular}{|l|l|l|}
\hline
\rowcolor{githublightgray}
Name & Derivative & Result\\
\hline
Inverse Sine&$\frac{d}{dx}[sin^{-1}u]$&$\frac{u\prime}{\sqrt{1-u^2}}$\\
\hline
Inverse Tangent&$\frac{d}{dx}[tan^{-1}u]$&$\frac{u\prime}{1+u^2}$\\
\hline
Inverse Secant&$\frac{d}{dx}[sec^{-1}u]$&$\frac{u\prime}{\&u\&\sqrt{u^2-1}}$\\
\hline
\end{tabular}
\end{table}

\subsection{Separable Differential Equation}
If $\frac{dy}{dx}=f(y)*g(x)$ than $\frac{1}{f(y)}dy=g(x)dx$\\
If $\frac{dy}{dt} = ky$ where k is a constant than $\lvert y\rvert=Ce^{kt}$\\
If $f(x)=x+y$ then $f\prime(x)=1+\frac{dy}{dx}$\\
\clearpage
\section{Integrals}
Area under the curve
\subsection{Basic Integrals}
Note that $F\prime(x)=f(x)$
\renewcommand{\arraystretch}{2}
\arrayrulecolor{lightgray}
\begin{table}[H]
\rowcolors{2}{githublightgray}{githubbackground}
\begin{tabular}{|l|l|l|}
\hline
\rowcolor{githublightgray}
Name & Integral & Result\\
\hline
Fundamental theorem of Calculus&$\int_{a}^{b}f(x)$&$F(b)-F(a)$\\
\hline
Simple Function & $\int f(x)$&$F(x)+C$\\
\hline
Power Rule & $\int [x^n]dx$&$\frac{x^{n+1}}{n+1}+C$\\
\hline
Reciprocal Function & $\int [\frac{1}{x}]dx$&$ln|x|+C$\\
\hline
Exponential Function & $\int [a^u]dx$&$\frac{a^u}{ln(a)u\prime}$\\
\hline
\end{tabular}
\end{table}

\subsection{Trigometric Integrals}
\renewcommand{\arraystretch}{2}
\arrayrulecolor{lightgray}
\begin{table}[H]
\rowcolors{2}{githublightgray}{githubbackground}
\begin{tabular}{|l|l|l|}
\hline
\rowcolor{githublightgray}
Name & Integral & Result\\
\hline
Sine & $\int[sinx]dx$ & $-cosx+C$\\
\hline
Cosine &$\int[cosx]dx$ & $sinx+C$\\
\hline
Secant &$\int[sec^2x]dx$&$tanx+C$\\
\hline
Cosecant &$\int[csc^2x]dx$&$-cotx+C$\\
\hline
Sectan &$\int[secxtanx]dx$&$secx+C$\\
\hline
Coseccotan&$\int[cscxcotx]dx$&$-cscx+C$\\
\hline
\end{tabular}
\end{table}
\subsection{Volumes of Revolution}
Rotating a function(s) around an axis to find the volume of the 3d resulting object
\renewcommand{\arraystretch}{2}
\arrayrulecolor{lightgray}
\begin{table}[H]
\rowcolors{2}{githublightgray}{githubbackground}
\begin{tabular}{|l|l|l|}
\hline
\rowcolor{githublightgray}
Name & Formula\\
\hline
Total Volume & $\int_{a}^{b}A(x)$\\
\hline
Disk Method &$\int_{a}^{b}\pi f(x)^2dx$s\\
\hline
Shell method &$\int_{a}^{b}2\pi xf(x)dx$\\
\hline
Shell between &$\int_{a}^{b}2\pi x(f(x)-g(x))dx$\\
\hline
Washer method &$\int_{a}^{b}\pi(f(x)^2-g(x)^2)dx$\\
\hline
\end{tabular}
\end{table}


\end{document}